\documentclass[]{article}
\usepackage{lmodern}
\usepackage{amssymb,amsmath}
\usepackage{ifxetex,ifluatex}
\usepackage{fixltx2e} % provides \textsubscript
\ifnum 0\ifxetex 1\fi\ifluatex 1\fi=0 % if pdftex
  \usepackage[T1]{fontenc}
  \usepackage[utf8]{inputenc}
\else % if luatex or xelatex
  \ifxetex
    \usepackage{mathspec}
  \else
    \usepackage{fontspec}
  \fi
  \defaultfontfeatures{Ligatures=TeX,Scale=MatchLowercase}
\fi
% use upquote if available, for straight quotes in verbatim environments
\IfFileExists{upquote.sty}{\usepackage{upquote}}{}
% use microtype if available
\IfFileExists{microtype.sty}{%
\usepackage{microtype}
\UseMicrotypeSet[protrusion]{basicmath} % disable protrusion for tt fonts
}{}
\usepackage[margin=1in]{geometry}
\usepackage{hyperref}
\hypersetup{unicode=true,
            pdftitle={Práctica Data Driven Security},
            pdfauthor={Grupo G: Yaneth Gonzalez, Miriam Jiménez, Anabel Vilchez, Alejandro Moreno},
            pdfborder={0 0 0},
            breaklinks=true}
\urlstyle{same}  % don't use monospace font for urls
\usepackage{graphicx,grffile}
\makeatletter
\def\maxwidth{\ifdim\Gin@nat@width>\linewidth\linewidth\else\Gin@nat@width\fi}
\def\maxheight{\ifdim\Gin@nat@height>\textheight\textheight\else\Gin@nat@height\fi}
\makeatother
% Scale images if necessary, so that they will not overflow the page
% margins by default, and it is still possible to overwrite the defaults
% using explicit options in \includegraphics[width, height, ...]{}
\setkeys{Gin}{width=\maxwidth,height=\maxheight,keepaspectratio}
\IfFileExists{parskip.sty}{%
\usepackage{parskip}
}{% else
\setlength{\parindent}{0pt}
\setlength{\parskip}{6pt plus 2pt minus 1pt}
}
\setlength{\emergencystretch}{3em}  % prevent overfull lines
\providecommand{\tightlist}{%
  \setlength{\itemsep}{0pt}\setlength{\parskip}{0pt}}
\setcounter{secnumdepth}{0}
% Redefines (sub)paragraphs to behave more like sections
\ifx\paragraph\undefined\else
\let\oldparagraph\paragraph
\renewcommand{\paragraph}[1]{\oldparagraph{#1}\mbox{}}
\fi
\ifx\subparagraph\undefined\else
\let\oldsubparagraph\subparagraph
\renewcommand{\subparagraph}[1]{\oldsubparagraph{#1}\mbox{}}
\fi

%%% Use protect on footnotes to avoid problems with footnotes in titles
\let\rmarkdownfootnote\footnote%
\def\footnote{\protect\rmarkdownfootnote}

%%% Change title format to be more compact
\usepackage{titling}

% Create subtitle command for use in maketitle
\newcommand{\subtitle}[1]{
  \posttitle{
    \begin{center}\large#1\end{center}
    }
}

\setlength{\droptitle}{-2em}
  \title{Práctica Data Driven Security}
  \pretitle{\vspace{\droptitle}\centering\huge}
  \posttitle{\par}
  \author{Grupo G: Yaneth Gonzalez, Miriam Jiménez, Anabel Vilchez, Alejandro
Moreno}
  \preauthor{\centering\large\emph}
  \postauthor{\par}
  \predate{\centering\large\emph}
  \postdate{\par}
  \date{2018-05-19}


\begin{document}
\maketitle

\section{Introducción}\label{introduccion}

El proposito de la práctica es evidenciar las posibles vulnerabilidades
industriales que pueden darse en diferenres paises. La pregunta que
intentamos responder es si \textbf{existe algun pais más propenso a
tener vulnerabilidades industriales}.

Para ello buscaremos en \href{http://www.shodan.io}{Shodan} todos los
dispositivos industriales públicos en Internet y lo cruzaremos con las
vulnerabilidades que tengan este tipo de modelos de dispositivos
industriales publicos en la \href{https://cve.mitre.org}{Common
Vulnerabilities and Exposures}. El resultado lo representaremos en un
mapa y realizaremos un analisis.

Shodan es un motor de búsqueda que permite al usuario encontrar iguales
o diferentes tipos específicos de equipos (routers, servidores, etc.)
conectados a Internet a través de una variedad de filtros

Los paquetes utilizados para la obtencion de las fuentes de datos y
calculos en R pueden encontrarse en:

\url{https://github.com/amperis/upc-master-datadriven}

Esta guía puede encontratse en:

\url{https://github.com/amperis/upc-master-datadriven-informe}

\section{Descripción de los modelos de
datos}\label{descripcion-de-los-modelos-de-datos}

Las fuentes de datos que utilizaremos en la práctica serán las
siguientes:

\begin{itemize}
\tightlist
\item
  Fuente de datos con dispositivos industriales según fabricantes
\item
  Fuente de datos Shodan de dispositivos industriales accesibles desde
  Internet
\item
  Fuente de datos de CVE
\end{itemize}

A continuación se describen las caracteristicas de cada una de las
fuentes de datos.

\subsection{Fuente de datos con dispositivos industriales según
fabricantes}\label{fuente-de-datos-con-dispositivos-industriales-segun-fabricantes}

Esta fuente de datos es un archivo CSV generado manualmente a través de
los diferentes dispositivos industriales más comunes actualmente. Nos
hemos basado en diferente documentacion técnica disponible en Internet
así como en la docuemntacion impartida en el master. Un ejemplo de esta
información se puede enconrar en la siguiente URL:

\url{https://cybersecuritylaboratory.wordpress.com/2016/10/05/utilizando-shodan-para-encontrar-sistemas-de-control-industrial}

Esta lista debe actualizarse a medida que se conozcan más fabricantes.
Las cabeceras de dicha tabla son las siguientes:

\begin{itemize}
\tightlist
\item
  farbricante: string con el numbre del fabricante del dispositivo
  industrial
\item
  producto: string con el modelo del producto industrial
\item
  cadena de busqueda: string de busqueda en Shodan
\end{itemize}

A continuación se muestran los 15 primeros regisotros de este CSV:

\begin{verbatim}
#> Loading required package: bitops
#>                  fabricante.producto.cadena_busqueda
#> 1                      Siemens;Scalance S;Scalance S
#> 2                      Rabbit;Generic;Z-World Rabbit
#> 3                          Siemens;Simatic S7;S7-300
#> 4                                 Trend;IQ3xcite;iq3
#> 5              Schneider Electric;Generic;PowerLogic
#> 6        Schneider Electric;PowerLogic ION;Meter ION
#> 7         Schneider Electric;Tac XENTA 913;TAC/Xenta
#> 8   Schneider Electric;Generic;Power Measurement Ltd
#> 9           Schneider Electric;Generic;Schneider-WEB
#> 10                             ABB;RTU500;ABB RTU560
#> 11                         ABB;Generic;ABB Webmodule
#> 12 Schneider Electric;PowerLogic PM;PowerLogic PM800
#> 13            General Electric;Proficy;ProficyPortal
#> 14        Schneider Electric;CitectSCADA;CitectSCADA
#> 15           Generic;Generic;openerp server:CherryPy
\end{verbatim}

Actualmente se disponen de:

\begin{verbatim}
#> un total de 87 dispositivos industriales
\end{verbatim}

\subsection{Fuente de datos Shodan de dispositivos industriales
accesibles desde
Internet}\label{fuente-de-datos-shodan-de-dispositivos-industriales-accesibles-desde-internet}

Esta fuente de datos es un archivo de datos CVE obtenida a través de un
proceso interativo programado en R desde el cual por cada dispositivo
industrial obtenido del punto anterior es buscado en Shodan. Esta
busqueda me devuelve diferentes información del dispositivo industrial
encontrado y publico en Internet.

Esta lista puede actualizarse desde el siguiente programa
disponible\ldots{}.

Las cabeceras del CVE obtenido son las siguientes:

\begin{itemize}
\tightlist
\item
  v\_fabricante: string con el nombre del fabricante del dispositovo
  industrial
\item
  v\_producto: string con el modelo del producto del dispositivo
  industrial
\item
  v\_cadena\_busqueda: string con la cadena de busqueda Shodan que se a
  utilizado para encontrar este dispositovo
\item
  aux.matches.ip\_str: string con la IP publica del dispositivo en
  formato ICANN de 32 bits
\item
  aux.matches.isp: string con el nombre comercial del ISP (Internet
  Service Provider) de dicha direccion IP
\item
  aux.matches.location.city: string de la ciudad donde está ubicada
  fisicamente dicha IP
\item
  aux.matches.location.longitude: string con la coordenada de longitud
  geografica
\item
  aux.matches.location.latitude: strinf con la coordenada de latitud
  geografica
\item
  aux.matches.location.country\_name: string con el pais
\item
  aux.matches.location.country\_code3: strinf con el pais en formato
  ISO-3166.1
\item
  aux.matches.os.au: string con el identificador del Sistema Autonomo al
  que pertenece la IP
\end{itemize}

A continuación se muestran los 15 primeros regisotros de este CSV:

\begin{verbatim}
#>    v_fabricante.v_producto.v_cadena_busqueda.aux.matches.ip_str.aux.matches.isp.aux.matches.location.city.aux.matches.location.longitude.aux.matches.location.latitude.aux.matches.location.country_name.aux.matches.location.country_code3.aux.matches.os.au ...
#> 1                                                                                                                             Rabbit;Generic;Z-World Rabbit;96.94.59.137;Comcast Cable;Jacksonville;-81.6188;30.2406;United States;USA;NA;Comcast Business;AS7922
#> 2                                                                                                                                                  Rabbit;Generic;Z-World Rabbit;86.45.60.230;Eircom;Westport;-9.51669999999999;53.8;Ireland;IRL;NA;Eircom;AS5466
#> 3                                                                                                                                              Rabbit;Generic;Z-World Rabbit;92.43.229.142;OmniAccess S.L.;NA;-3.684;40.4172;Spain;ESP;NA;OmniAccess S.L.;AS44431
#> 4                                                                                                                                   Rabbit;Generic;Z-World Rabbit;91.100.45.1;Dansk Kabel TV;Vejle;9.53569999999999;55.7093;Denmark;DNK;NA;Dansk Kabel TV;AS15516
#> 5                                                                       Rabbit;Generic;Z-World Rabbit;166.236.229.130;Wireless Data Service Provider Corporation;NA;-97.822;37.751;United States;USA;NA;United States Cellular Telephone Company (greater;AS18933
#> 6                                                                                                                           Rabbit;Generic;Z-World Rabbit;24.153.220.218;Time Warner Cable;Dallas;-96.7459;32.8731;United States;USA;NA;Time Warner Cable;AS11427
#> 7                                                                                                                                             Rabbit;Generic;Z-World Rabbit;86.190.210.167;BT;Stornoway;-6.36670000000001;58.2167;United Kingdom;GBR;NA;BT;AS2856
#> 8                                                                                                                                        Rabbit;Generic;Z-World Rabbit;62.133.120.207;KPN Mobile;NA;4.89949999999999;52.3824;Netherlands;NLD;NA;KPN Mobile;AS8737
#> 9                                                                                                                                 Rabbit;Generic;Z-World Rabbit;67.238.58.178;CenturyLink;Port Charlotte;-82.1266;27.0228;United States;USA;NA;CenturyLink;AS2379
#> 10                                                                                                                                             Rabbit;Generic;Z-World Rabbit;69.158.210.124;Bell Mobility;NA;-79.3716;43.6319;Canada;CAN;NA;Bell Mobility;AS36522
#> 11                                                                                                                               Rabbit;Generic;Z-World Rabbit;70.90.68.44;Comcast Cable;North Port;-82.2552;27.0633;United States;USA;NA;Comcast Business;AS7922
#> 12                                                                                                                                                         Rabbit;Generic;Z-World Rabbit;186.155.199.112;ETB;Medellin;-75.5636;6.2518;Colombia;COL;NA;ETB;AS19429
#> 13                                                                                                                                                  Rabbit;Generic;Z-World Rabbit;84.253.136.220;Mc-link SpA;Rome;12.5113;41.8919;Italy;ITA;NA;Mc-link SpA;AS5396
#> 14                                                                                                                                 Rabbit;Generic;Z-World Rabbit;80.229.184.115;Plusnet;Haringey;-0.0833000000000084;51.5833;United Kingdom;GBR;NA;Plusnet;AS6871
#> 15                                                                                                                Rabbit;Generic;Z-World Rabbit;31.148.230.2;Olivenet Network S.L.;Marbella;-4.88579999999999;36.5154;Spain;ESP;NA;Olivenet Network S.L.;AS201746
\end{verbatim}

Actualmente se disponen de:

\begin{verbatim}
#> un total de 5289 dispositivos industriales
\end{verbatim}

\subsection{Fuente de datos de CVE}\label{fuente-de-datos-de-cve}

\section{Objetivos del análisis de
datos}\label{objetivos-del-analisis-de-datos}

\section{Análisis de datos}\label{analisis-de-datos}

\section{Conclusiones}\label{conclusiones}

\section{Referencias utilizadas}\label{referencias-utilizadas}

\begin{itemize}
\tightlist
\item
  (\url{https://mauriciogtec.github.io/rGallery/entries/tutoriales/crear_paquetes/crear_paquete.html})
\item
  (\url{https://www.rstudio.com/wp-content/uploads/2015/02/rmarkdown-cheatsheet.pdf})
\end{itemize}


\end{document}
